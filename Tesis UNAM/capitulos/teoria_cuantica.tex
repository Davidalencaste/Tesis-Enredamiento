\documentclass[../Main.tex]{subfiles}

\begin{document}

La descripción de cualquier sistema físico requiere, al menos, la identificación de los siguientes cinco elementos:
\begin{enumerate}
    \item Estados
    \item Observables
    \item Función de probabilidad
    \item Ecuación de evolución
    \item Regla de composición entre sistemas
\end{enumerate}

Por ejemplo, en la \textit{mecánica clásica}, el estado de un sistema está determinado por un punto en el espacio fase, mientras que las observables se definen como funciones sobre dicho espacio. Por otra parte, si el sistema está formado por un número grande de partículas, el estado se describe mediante una densidad de probabilidad, esto es, una función normalizada en el espacio fase.

La evolución de los estados se realiza mediante la dinámica Hamiltoniana, la cual preserva no solo la energía, sino también los volúmenes en el espacio fase (Teorema de Liouville), esto garantiza la conservación de la normalización de las funciones de probabilidad. Por último, el sistema compuesto de dos subsistemas tiene como espacio fase el producto cartesiano de los espacios fase de los sistemas individuales.\\



El objetivo de esta sección es mostrar que estos elementos también están presentes en la teoría cuántica, de hecho, muchos de estos se encuentran definidos en los llamados axiomas de la mecánica cuántica.






\section{Espacio de Estados}
Tradicionalmente, el primer postulado de la mecánica cuántica define a los estados de la siguiente manera.\\
\textit{El estado de un sistema físico esta definido por un vector de estado, que es un elemento de un espacio de Hilbert} $\mathcal{H}$ \cite{Cohen}.
Donde cada un vector de estado (el vector asociado al estado $\psi$) se asocia a un ket $\ket{\;}$
\begin{equation}
    \textbf{Estado cuántico} \equiv \ket{\psi}.
\end{equation}
La dimensión del espacio de estados es igual al número de kets linealmente independientes por definición. Si el número de kets linealmente independientes es infinito, decimos que el espacio de estados tiene una dimensión infinita.\\

De aquí en adelante nos enfocaremos en el caso de sistemas cuánticos con dimensión finita. Entonces, para este caso los kets forman un espacio vectorial lineal, donde cualquier combinación lineal de kets es, de igual manera, un ket. 
Mediante esta definición, se dice que los kets de un ensamble son linealmente independientes si ninguno de ellos puede ser expresado como una combinación lineal de otros \cite{Messiah}.\\


Es bien sabido, que cada espacio vectorial puede ser asociado con un espacio vectorial dual, por tanto, existe la asociación a un \textit{espacio vectorial dual del espacio de kets}.


Consideremos un función lineal $\chi(\ket{\psi})$ que asocia un número complejo con cada ket $\ket{\psi}$, es decir, la función $\chi(\ket{\psi})$ define el bra $\bra{\chi}$ \cite{Cohen}, tal que
\[
\chi: \mathcal{H} \to \mathbb{C}: | \psi \rangle \mapsto \langle \chi | \psi \rangle.
\]


El conjunto de funciones lineales definido sobre los kets constituye también a un espacio vectorial, el cual es llamado \textit{espacio dual} de $\mathcal{H}$ y es denotado por $\mathcal{H}^*$ \cite{Cohen}.
  

Con la idea de introducir el producto escalar en $\mathcal{H}$, se puede observar que es posible asociar a cada ket un bra. De esta manera, la acción lineal $\chi:\mathcal{H}\to \mathbb{C}$ se puede expresar como
\begin{equation}
\chi(\ket{\psi})\equiv\left(\ket{\chi},\ket{\psi}\right),
\end{equation}
donde $\left(\cdot,\cdot\right)$ representa el producto interno en $\mathcal{H}$ \cite{Cohen}. Así, el producto interno en términos de kets y bras \cite{Messiah}:
\begin{equation}
    \ip{\psi}{\varphi}\equiv\left(\ket{\psi},\ket{\varphi}\right),
\end{equation}
el cual, satisface las siguientes condiciones:
\begin{itemize}
    \item $\ip{\psi}{\varphi}=\overline{\ip{\varphi}{\psi}}$,
    \item $\ip{(\alpha\psi+\beta\varphi)}{\phi}=\alpha\ip{\psi}{\phi}+\beta\ip{\varphi}{\phi}$,
    \item $\ip{\psi}\geq 0$,
    \item Si $\ip{\psi}=0$, entonces $\ket{\psi}=0$.
\end{itemize}


En el caso de espacios de dimensión infinita, además de cumplir las propiedades del producto interno, es necesario que toda sucesión de Cauchy converja a un elemento del propio espacio; es decir, el espacio debe ser \textit{completo}. Asimismo, se requiere que el espacio admita una base ortonormal numerable. A los espacios que satisfacen estas propiedades se les denomina \textit{espacios separables} \cite{Messiah}.

%\subsection{Espacios de Hilbert}


\begin{comment}
A continuación se muestran dos ejemplos fundamentales en la mecánica cuántica de espacios de Hilbert.
\begin{enumerate}
    \item \textbf{Funciones cuadrático integrables}\\
    Funciones $\psi(q_1,q_2,\dots,q_n$) tales que:
    \[\int \abs{\psi(q_1\dots q_n)}^2d\tau < \infty\]
    Son funciones cuadrático integrables con producto interno definido como \[\left(\psi,\phi\right)=\int \bar{\psi}(q_1,\dots,q_n)\phi(q_1,\dots,q_n)d\tau\]
    Este espacio posee, la propiedad de ser completo, cumpliendo así las propiedades de espacio de Hilbert\cite{Messiah}.
    \item \textbf{Espacio del Qubit}\\
    Un qubit se describe mediante sistemas compuestos formados por estados individuales de dos niveles\cite{Kaye}. Se tiene un espacio discreto 2-dimensional. Generalmente se le asocia a este espacio la \textit{base computacional}:
    \begin{equation}
        \{\ket{0},\ket{1}\}.
    \end{equation}
\end{enumerate}

    Nótese la analogía con la computación clásica, donde tenemos dos posibles valores para los bits $0$ ó $1$. Los qubits diferenciándose, como se verá mas adelante, en la superposición de ambos estados. 
    Algunos sistemas que se usan para modelar qubits, y por tanto, siendo sistemas con dos niveles de energía se muestran a continuación.
    \begin{itemize}
        \item Estado de espín de partículas con espín $1/2$. Siendo la base computacional: $\{\textit{Spin-down}$, $\textit{Spin-up}\}$
        \item Electrón orbitando núcleos con una separación energética grande entre el segundo nivel energético y el tercero para confinar al electrón unicamente a los dos primero niveles energéticos del átomo \cite{Kaye}. Con base computacional: $\{E_0, E_1\}$, los niveles energéticos donde se encuentra la partícula.
        \item Polarización de un fotón, con base computacional: $\{\Bar{V}, \Bar{H}\}$ la polarización vertical y horizontal.
    \end{itemize}
\end{comment}


\subsection{Espacio de Rayo}

Para definir el espacio de rayos es necesario notar que una medición completa
en mecánica cuántica no da como resultado un vector específico del espacio de
Hilbert, sino una clase de equivalencia de vectores relacionada mediante la
multiplicación por un número complejo no nulo. En otras palabras, sobre el
espacio de Hilbert existe una acción natural del grupo abeliano
$\C_0 := \C - \{0\}$ dada por
\begin{equation}
    \ket{\psi} \mapsto \lambda \ket{\psi}
    = \varrho \e^{i \theta} \ket{\psi},
    \qquad
    \varrho \in \R^{+}, \ \theta \in [0,2\pi).
\end{equation}

Al fijar el módulo del factor complejo $\varrho$ (factor de escalamiento) se elimina la
libertad asociada a la reescalación real y se obtiene la llamada esfera de
estados normalizados
\begin{equation}
    S^{2n-1}
    :=
    \left\{
        \ket{\psi} \in \H_0
        \,\middle|\,
        \ip{\psi} = 1
    \right\},
\end{equation}
donde $\mathcal{H}_0=\mathcal{H}-\{0\}$. De esta forma, los estados físicos se identifican con las órbitas de la acción del grupo $U(1)$ sobre la esfera de estados normalizados $S^{2n-1}$; es decir, con circunferencias $S^1$ embebidas en $S^{2n-1}$.\\


Por otra parte, al identificar estados que difieren únicamente por una fase
global, $\ket{\psi} \sim \e^{i\theta}\ket{\psi}$, siendo estas, orbitas de acción de $U(1)$, se obtiene el espacio
proyectivo de Hilbert, definido como el conjunto de todos los rayos del espacio
de Hilbert,
\begin{equation}
    \Proj(\H_0)
    := \{\lambda\ket{\psi}\;|\; \lambda\in \C_0\}
\end{equation}

El siguiente diagrama resume las relaciones entre el espacio de Hilbert,
la esfera de estados normalizados y el espacio proyectivo.

\[
\begin{tikzcd}[column sep=large, row sep=large]
\H_0
\arrow[dd]
\arrow[rd] 
&  \\
& S^{2n-1}
\arrow[ld] \\
\Proj(\H_0)
\end{tikzcd}
\]



El espacio $\Proj(\H_0)$ es conocido como el \emph{espacio proyectivo de Hilbert}.
En el caso de sistemas de dimensión finita, este espacio se denomina
\emph{espacio complejo proyectivo } y se denota por $\C \Proj(\H_0)$. Por lo tanto, el espacio proyectivo complejo proporciona una descripción del
espacio de estados físicos en la cual se han eliminado las redundancias
asociadas a la norma y a la fase global.


Es importante mencionar que no es usual trabajar directamente con sistemas cuánticos en el espacio proyectivo $\Proj(\H_0)$, ya que este no admite coordenadas globales. No obstante, dicho espacio posee propiedades geométricas de gran relevancia, entre las que destaca la existencia de la métrica de Fubini-Study. Además, como se mostrará en la siguiente sección, existe una correspondencia uno a uno entre los elementos del espacio proyectivo y las matrices densidad de rango uno.


\subsection{Espacio de Operadores de Densidad}
Es común que se estudie la mecánica cuántica en el espacio de Hilbert, donde describimos al estado con un ket $\ket{\psi}$, pero como se mostrará a continuación, esta manera de representar los estados limita en gran medida el tipo de sistemas con los que podemos trabajar.\\
Cuando se considera un sistema representado por un ket, se dice indirectamente que el sistema físico esta caracterizado por un mismo tipo de estado $\ket{\psi}$. Un ejemplo de ello es cualquier experimento tipo \textit{filtro}, en los cuales el filtro arroja un resultado completamente predecible con probabilidad $100\%$ de ocurrir y colapsando siempre a un único estado; un haz de luz filtrado a través de un de un prisma de Nicol es un ejemplo de ello \cite{Fano}. Con este tipo de sistemas, realizar una medición  $\expval{A}$ es esencialmente sencillo, pues el resultado obtenido es seguro que proviene de un mismo tipo de estado, de la misma naturaleza. 

Por lo tanto, tenemos una manera matemática de describir este tipo de sistemas, pero ¿Qué ocurre con el haz de luz antes de pasar por el polarizador?; en otras palabras, ¿Cómo describir la luz no polarizada?. En este caso, la luz puede describirse como una mezcla de polariazadores; es decir, una mezcla ponderada de estados puros donde el haz de luz esta completamente caracterizado por los i-esimos filtros. A este tipo de sistemas se les conoce como \textit{estados mixtos}. En ellos, realizar una medición al sistema implica una mezcla ponderada de las mediciones individuales en cada filtro, como se muestra a continuación.
\begin{equation}
    \expval{A}=\sum_i w_i\expval{A}_i,
\end{equation}
donde $\omega_i$ representa el peso estadistico del i-esimo filtro.\\ Realizando ciertas manipulaciones:

\begin{equation}
\expval{A}=\sum_i w_i\expval{A}_i = \sum_{n,n'}  Q_{nn'}\sum_i w_i c^{i\,*}_{n'}c^i_n
\label{eq:observable_estadistico}
\end{equation}

Definiendo la \textit{matriz de densidad}:
\begin{equation}
    \rho_{nn'}=\sum_i w_i\, c^{i\,*}_{n'}\,c^i_n
\end{equation}
De esta manera, ~\eqref{eq:observable_estadistico} pasa a ser:
\begin{equation}
    \expval{A}= \sum_{nn'}A_{nn'}\rho_{nn'}=\sum_{n'}(A\rho)_{nn'}=\Tr(A\rho)
    \label{def:Operador_densidad}
\end{equation}
La ecuación ~\eqref{eq:Operador_densidad} nos define el \textbf{Operador de densidad}. Es importante notar que este operador no depende de la base con la que se trabaje, y esto resulta obvio usando el ejemplo de luz no polarizada, mencionamos que puede ser vista como mezcla de polarizadores, pero bien pueden ser polarizadores lineales, como circulares \cite{Fano}. Es por ello que aparece de forma natural la traza, la cual es una operación independiente de la base en la que se trabaje. \\ Algunos textos definen el operador densidad de la siguiente forma \cite{Sakurai2011}:
\begin{equation}
    \rho\equiv \sum_i w_i\op{\phi_i}.
\end{equation}
Esta expresión cumple con la misma definición ~\eqref{eq:Operador_densidad}, en realidad son equivalentes, sin embargo esta definición podría dar la interpretación en los estados mixtos como mezcla de estados puros $\ket{\phi}$, pero como ya se comento anteriormente, esto simplemente es por la base de operadores $\{\op{\alpha}\}$ escogida, donde la definición es completamente valida para cualquier base.\\ Una propiedad importante del operador densidad es la \textit{propiedad de normalización}, que viene del hecho que la suma de los pesos $w_i$ es $1$ por construcción estadística.
\begin{equation}
    \Tr(\rho)= \sum_n \rho_{nn}=\sum_i w_i = 1.
\end{equation}

Ahora, nos gustaría tener una clase de medida que cuantifique que tan puro o mixto es un estado por medio del operador de densidad, de tal manera que resulte evidente conocer con que tipo de ensamble se esta trabajando por medio del operador de densidad. \\Una medida de lo anterior es la traza de $\rho^2$. Note en un estado consta por un $w_i=1$ para algún $\ket{\alpha_i}$ y $w_i=0$ para cualquier otro estado. De esta manera, el operador densidad de un estado puro es de la siguiente forma:
\begin{equation}
    \rho = \op{\alpha}.
\end{equation}
Es directo notar que para este estado:
\begin{equation}
    \rho^2=\rho,
\end{equation}
concluyendo, para un ensamble puro se tiene
\begin{equation}
    \Tr(\rho^2)=1.
\end{equation}
Además se tiene que para cualquier ensamble mixto, $0\le\Tr(\rho^2)\le1$. De esta manera, la traza de $\rho^2$ nos cuantifica que tan mixto o puro es un ensamble.\\ Esta medida es sencilla de calcular, sin embargo nos gustaría una medida que dependa de alguna cantidad física medible. Con esta motivación en mente, vamos a considerar un ensamble de partículas con espín $\frac{1}{2}$ y espín promedio definido (cantidad física medible), es decir, vamos a considerar un ensamble con dos niveles de energía.\\ Hacemos esta consideración, pues el desarrollo de esta tesis se basa en qubits que es un sistema de dos niveles de energía.\\

Dado a \eqref{eq:Operador_densidad} definimos el espín promedio del ensamble como sigue:
\begin{equation}
    \bm{S}=\Tr(\rho \bm{\sigma}).
\end{equation}
Donde $\bm{\sigma}=(\sigma_x,\sigma_y,\sigma_z)$ son las matrices de Pauli.\\ 

Definimos una cantidad que mida de manera cuantitativa el desorden del sistema, la cual dependa del tipo de ensamble. Exigimos que dicha cantidad sea mínima (cero) para un ensamble puro y máxima para un ensamble completamente aleatorio.  

En termodinámica, la magnitud que cumple con estas propiedades es la \textit{entropía}. Motivados por esto, definimos la \textit{entropía en mecánica cuántica estadística} como \cite{Sakurai2011}
\begin{equation}
    S_{entropia} = -k_B \Tr(\rho \ln \rho),
    \label{def:entropia}
\end{equation}
donde $k_B$ es la constante de Boltzmann y $\rho$ es el operador densidad del sistema.\\
A continuación, buscaremos el operador densidad $\rho$ que describe el equilibrio térmico de un sistema con espín definido.

Consideremos la acción $F$:
\begin{align}
    F
    &=\Tr(S_{entropia}+\alpha_x S_x+\alpha_y S_y+\alpha_z S_z+\lambda \rho)\\
    &=\Tr(-\rho\ln{\rho}+\bm{\alpha}\cdot\Bar{S}+\lambda\rho).
\end{align}
Donde $S_i$ es el valor promedio de la i-esima componente del espín, es decir, $S_i=\Tr(\rho \sigma_i)$.\\
Sabemos que la entropía debe maximizarse en un estado de equilibrio, y dado que la entropía es cóncava en $\rho$ basta con obtener el punto donde $\delta F = 0$ para encontrar el estado $\rho$ que maximiza la entropía.
\begin{align}
0=\delta F
&= \Tr(\ln{\rho}+\Id+\bm{\alpha}\cdot \bm{\sigma}+\lambda\Id)\delta\rho.
\label{eq:multiplicadores}
\end{align}
entonces al ser para cualquier variación, necesariamente se requiere que:
\begin{equation}
    \rho = \frac{\e^{-\bm{\alpha}\cdot \bm{\sigma}}}{\e^{(1+\lambda)\Id}}.
    \label{eq:rho_wo_norm}
\end{equation}
Usando la restricción $\Tr(\rho)=1$, se obtiene
\begin{equation}
    \e^{(1+\lambda)\Id}=\Tr(\e^{-\bm{\alpha}\cdot\bm{\sigma}}).
    \label{eq:rho's_normaliz}
\end{equation}
Sustituyendo ~\eqref{eq:rho's_normaliz} en ~\eqref{eq:rho_wo_norm} se obtiene el operador densidad de un sistema con espín $\bm{S}$ y en equilibrio térmico.
\begin{equation}
    \rho = \frac{\e^{-\bm{\alpha}\cdot\bm{\sigma}}}{\Tr(\e^{-\bm{\alpha}\cdot\bm{\sigma}})}.
    \label{eq:rho_semifinal}
\end{equation}
Solo es necesario ahora incluir la restricción de espín para despejar los valores $\alpha_i$. Primero, recordando que:
\begin{equation}
(\bm{\sigma}\cdot\bm{\alpha})^{m}
=
\begin{cases}
|\bm{\alpha}|^{m} \;\Id,
& m \ \text{par}, \\[4pt]
|\bm{\alpha}|^{m}\,
\bm{\sigma}\cdot\hat{\bm{n}},
& m \ \text{impar},
\end{cases}
\end{equation}
Con ello, expandimos en serie de Taylor la exponencial:
\begin{align}
\e^{-\bm{\alpha}\cdot\bm{\sigma}}
&=
\sum_{m=0}^{\infty}
\frac{(-1)^m}{m!}
(\bm{\alpha}\cdot\bm{\sigma})^{m}
\notag \\
&=
\sum_{k=0}^{\infty}
\frac{|\bm{\alpha}|^{2k}}{(2k)!}\,\Id
-
\sum_{k=0}^{\infty}
\frac{|\bm{\alpha}|^{2k+1}}{(2k+1)!}
(\bm{\sigma}\cdot\hat{\bm{n}})
\notag \\
&=
\cosh|\bm{\alpha}|\,\Id
-
\frac{\sinh|\bm{\alpha}|}{|\bm{\alpha}|}
(\bm{\alpha}\cdot\bm{\sigma}).
\label{eq:expansion_exponencial}
\end{align}
Con este resultado, aplicamos la traza a la exponencial:
\begin{align}
\Tr\!\left(\e^{-\bm{\alpha}\cdot\bm{\sigma}}\right)
&=
\Tr\!\left(
\cosh|\bm{\alpha}|\,\Id
-
\frac{\sinh|\bm{\alpha}|}{|\bm{\alpha}|}
(\bm{\alpha}\cdot\bm{\sigma})
\right)
\notag \\
&=
\cosh|\bm{\alpha}|\,\Tr(\Id)
-
\frac{\sinh|\bm{\alpha}|}{|\bm{\alpha}|}
\Tr(\bm{\alpha}\cdot\bm{\sigma})
\notag \\
&=
2\cosh|\bm{\alpha}|
-
\frac{\sinh|\bm{\alpha}|}{|\bm{\alpha}|}
\sum_i \alpha_i\; \cancelto{0}{\Tr(\sigma_i)}
\notag \\
&=
2\cosh|\bm{\alpha}|.
\label{eq:normalizacion_traza}
\end{align}
Sustituyendo ~\eqref{eq:normalizacion_traza} en ~\eqref{eq:rho_semifinal} y reacomodando:
\begin{equation}
    \rho = \frac{1}{2}\left(\Id - \frac{\tanh{|\bm{\alpha}|}}{|\bm{\alpha}|}\right)\bm{\alpha}\cdot\bm{\sigma}
    \label{eq:rho_semi_semifinal}
\end{equation}
Ahora, observando que
\begin{align}
    S_k 
    &= \Tr(\rho\,\sigma_k)
    = \Tr\!\left[\frac{1}{2}\left(\Id - \frac{\tanh |\bm{\alpha}|}{|\bm{\alpha}|}\sum_i \alpha_i\,\sigma_i \right)\sigma_k\right]
    \notag \\
    &= \frac{1}{2}\left(\cancelto{0}{\Tr(\sigma_k)}
    - \frac{\tanh |\bm{\alpha}|}{|\bm{\alpha}|}
    \sum_i \alpha_i \Tr(\cancelto{2\delta_{ik}}{\sigma_i\sigma_k}\hspace{.9cm})\right)
    &= -\frac{\tanh |\bm{\alpha}|}{|\bm{\alpha}|}\,\alpha_k ,
\end{align}
y sustituyendo este resultado en la ecuación ~\eqref{eq:rho_semi_semifinal}, se obtiene finalmente
\begin{equation}
\rho = \frac{1}{2}\left(\Id + \bm{S}\cdot \bm{\sigma}\right).
\label{eq:rho_final}
\end{equation}
Ahora, calculemos $\rho^2$ para hayar una relación con $\bm{S}$.
\begin{align*}
    \rho^2&=
    \frac{1}{4}\left(\Id + \bm{S}\cdot \bm{\sigma}\right)^2
    \notag \\
    &=\frac{1}{4}\left(\Id+2\bm{S}\cdot\bm{\sigma}+|\bm{S}|^2\,\Id\right).
\end{align*}
Calculando la traza de $\rho^2$:
\begin{align}
    \Tr(\rho^2)&=
    \Tr[\frac{1}{4}\left(\Id+2\,\bm{S}\cdot\bm{\sigma}+\bm{|S|}^2\,\Id\right)]
    \notag \\
    &= \frac{1}{4}\left[\cancelto{2}{\Tr(\Id)}+2\sum_iS_i\;\cancelto{0}{\Tr(\sigma_i)}\;+2|\bm{S}|^2\right]
    \notag \\
    &=\frac{1}{2}\left(1+|\bm{S}|^2\right).
\end{align}

De la última expresión se concluye que:

\begin{enumerate}
    \item Un estado puro corresponde a un vector de Bloch localizado en la superficie de la esfera de Bloch, es decir, $|\bm{S}|=1$.
    \item Un estado mixto corresponde a un vector de Bloch ubicado en el interior de la esfera, con $|\bm{S}|<1$.
    \item El estado completamente mezclado corresponde al caso $|\bm{S}|=0$, es decir, al origen de la esfera de Bloch.
\end{enumerate}

De esta manera, el espacio de operadores densidad es el conjunto de todas las $\rho$ de la forma de \eqref{eq:rho_final} y cada estado es identificado por su espín promedio $\bm{S}$ del sistema.
\begin{equation}
    \rho_{space}\equiv\left\{\rho=\frac{1}{2}\left(\Id + \bm{S}\cdot \bm{\sigma}\right)\;|\;\bm{S}\in \R^3 \right\}
\end{equation}


\begin{comment}
    La pregunta natural para cualquier formalismo es ¿Como hacer mediciones?. En este caso, al ser posibles los ensambles mixtos, se debe reflejar la medición de cada sub-estado en la medición del observable. Buscando este proposito se define el valor esperado de un observable $\Oper{\sigma}$ como:
\begin{equation}
    \expval{\Oper{\sigma}}=\sum_i\omega_i\expval{\Oper{\sigma}}{\psi_i}
    \label{eq:def_expval}
\end{equation}
De la Ecuación ~\eqref{eq:def_expval}:

\begin{align}
\sum_i \omega_i \expval{\Oper{\sigma}}{\psi_i}
    &= \sum_i \sum_a \sum_b 
       \omega_i \ip{\psi_i}{a}\mel{a}{\Oper{\sigma}}{b}\ip{b}{\psi_i} \\[4pt]
    &= \sum_{i,a,b} \omega_i \mel{a}{\Oper{\sigma}}{b}\ip{b}{\psi_i} \\[4pt]
    &= \Tr{\sum_{i,b}\omega_i \Oper{\sigma}\dyad{b}\dyad{\psi_i}} \\[4pt]
    &= \Tr{\sum_b\dyad{b}\rho} \\[4pt]
    &= \Tr{\Oper{\sigma}\rho}.
\end{align}
Obteniendo al final:
\begin{equation}
    \expval{\Oper{\Oper{\sigma}}}=\Tr{\Oper{\sigma}\rho}
\end{equation}
\end{comment}





\begin{comment}
    Lo único faltante para que ese formalismo describa cualquier sistema cuántico es una parte crucial de la mecánica cuántica, la superposición. ¿Cómo se describe la superposición de estados usando el operador de densidad.?

Es facil notar que el siguiente operador:
\begin{equation}
    \rho = \cos^2(\theta)\rho_1+\sin^2(\theta)\rho_2
\end{equation}
es de un estado mixto pues
\[\Tr{\rho^2}=1+2\cos^2(\theta)sin^2(\theta)\Tr{\rho_1\rho_2}
\]
donde el segundo termino no necesariamente es $0$. Así que se requiere otra forma de formular la superposición de estados mediante operadores de densidad de tal forma que se obtenga un estado puro.

\end{comment}


\section{Observables}
El concepto de medición es una parte fundamental de la física, es la herramienta que tenemos para obtener información un sistema mediante la manipulación del mismo por medio de una operación, llamemos $\bm{A}$.
De esta manera, al aplicar la operación $\bm{A}$ al sistema, obtendremos un valor de medición $a$. Esta es la idea general de medir en física.\\ Sin embargo, en mecanica cuántica la medición conlleva una característica adicional. \textit{Realizar una medición altera el estado del sistema} \cite{Sakurai2011}. Es decir, sea el estado inicial $\psi$, al realizar una medición $\bm{A}$ el estado colapsa a otro estado $\psi'$.
\begin{equation}
   \ket{\psi} \xrightarrow{\text{medición de } \bm{A}}\ket{\psi'}
   \label{diag: medicion}
\end{equation}

Esta operación $\bm{A}$es en mecanica cuantica un operador lineal que es una correspondencia lineal entre dos estados:
\begin{equation}
    \ket{\psi'}=\mathbf{A}\ket{\psi}.
\end{equation}
Donde dice que dos operadores $\bm{A}$ y $\bm{B}$ son iguales si:
\begin{equation}
    \expval{\bf{A}}{\psi}=\expval{\bf{B}}{\psi}.
\end{equation}


De igual forma como en la corresponecia uno a uno entre kets y bras, existe una relación antilineal en los operdadores.\\
Sea $\ket{v}$ el ket conjugado del bra $\bra{u}\bm{A}$. De esta manera, $\ket{v}$ depende antilinealmente del bra $\ket{u}$. Es por tanto una función lineal de $\ket{u}$. Esta correspondencia lineal define un operdador lineal por el nombre de \textit{conjugado de} $\bm{A}$, denotado por $\bm{A}^\dagger$.
\begin{equation}
    \ket{v}=\bm{A}^\dagger\ket{u}.
    \label{def: operador_conjugado}
\end{equation}



Sin embargo, no todos los operadores lineales tienen una interpretación física con la cual asociar una medición. Al tipo de operadores que tienen una interpretación física se les conoce como \textit{observables} que son operadores Hermitianos. A esto se le conoce habitualmente como el segundo postulado de la mecanica cuántica.\\

\textit{Toda cantidad medible se describe por un operador} $\bm{A}:\H \to \H$ que es \textbf{Hermitiano}; \textit{a esto se le llama un observable} \cite{Cohen}.\\

Un operador hermitiano $\bm{A}$ es aquel operador que es su mismo auto adjunto, denotado por "$\dagger$".

\begin{equation}
    \bm{A}^\dagger = \bm{A}.
\end{equation}

Esta definición conduce inmediatamente al postulado central de la medición, usualmente nombrado como el tercero, que afirma lo siguiente:\\

\textit{El único posible resultado de la medición de un observable $\bm{A}$ es uno de los eigenvalores de $\bm{A}$} \cite{Cohen}.\\


Consideremos un estado cuántico discreto general $\ket{\psi}$, expresado en la base formada por los autovectores de un operador observable $\bm A$ con espectro discreto, tal que
\[
\bm A \ket{e_i} = a_i \ket{e_i}.
\]
En esta base, el estado puede escribirse como
\begin{equation}
    \ket{\psi}=\sum_i c_i \ket{e_i},
    \label{eq: estado_general}
\end{equation}
donde los coeficientes $c_i$ representan las amplitudes de probabilidad asociadas a cada autovector.

La probabilidad de obtener el resultado $a_i$ al realizar una medición del observable $\bm A$ sobre el estado $\ket{\psi}$ está dada por \cite{Cohen}
\begin{equation}
    P(a_i)=\abs{\ip{e_i}{\psi}}^2=\abs{c_i}^2.
    \label{def: probabilidad_estado}
\end{equation}

Finalmente, para describir la naturaleza del proceso de medición, el estado posterior a la medición —suponiendo que se obtiene el resultado $a_i$— es
\begin{equation}
    \ket{\psi'}=
    \frac{\bm A \ket{\psi}}
    {\norm{\bm A \ket{\psi}}},
    \label{eq: colapso_estado}
\end{equation}
el cual corresponde, tras la normalización, al autovector de $\bm A$ asociado al autovalor obtenido en la medición.\\

Puede haber notado que este desarrollo se realizo usando el esquema de vectores del espacio de Hilbert, y no en el espacio de operadores de densidad. Sin embargo note, que en la definición \eqref{def:Operador_densidad}
del operador densidad yace en el concepto de medición. Lo unico nuevo es como ya vimos, la restricción de operadores que son medibles, los \textit{observables}.\\
Falta por conocer la forma del colapso de estado mencionada en la ecuación \ref{eq: colapso_estado} para operadores de densidad.\\

\begin{equation}
   \rho' \xrightarrow{\text{medición de } \bm{A}}\rho
   \label{diag: medicion}
\end{equation}

Tenemos de la ecuación \eqref{eq: colapso_estado} la forma general del colpaso de un estado dada una medición $\bm{A}$. De esta, se puede obtener su conjugado, dado \eqref{def: operador_conjugado}:
\begin{equation}
    \bra{\psi'}=\frac{\bra{\psi}\bm{A}^\dagger}{\norm{\bm{A}\ket{\psi}}}
\end{equation}
De esta manera, cada $\rho'$ de un estado puro puede escribirse como:
\begin{equation}
    \rho'=\frac{1}{\norm{\bm{A}\ket{\psi}}^2}\bm{A}\rho\bm{A}^\dagger
\end{equation}




\section{Evolución Temporal}
Es una prueba de como veo $gsdnf$
\begin{equation}
    \ket{\psi}=\ket{\phi}
\end{equation}
\section{Distribución de Probabilidad}
\section{Composición de Sistemas}

\end{document}
