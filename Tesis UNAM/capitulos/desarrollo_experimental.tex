\documentclass[../Main.tex]{subfiles}

\begin{document}

En las \cref{fig:example_image,fig:example_missingfigure} observamos que\dots mientras que en la \cref{tab:datos} se muestran los\dots

\begin{figure}
	\centering
	\includegraphics[height=5cm]{figExample}
	\caption{Pie de imagen.}
	\label{fig:example_image}
\end{figure}

\begin{figure}
	\centering
	\missingfigure{descripción de la imagen faltante.}
	\caption{Pie de imagen.}
	\label{fig:example_missingfigure}
\end{figure}

%INICIO TABLA DE DATOS———————
\DTLloaddb{data}{tablas/data.csv} % Cargar datos para la tabla

\begin{table}
	\caption{Encabezado de la tabla de datos.}
	\centering
\begin{tabular}{
S[parse-numbers=false, table-format = 1.3] % Configuración de formato de las columnas
S[parse-numbers=false, table-format = 2.4] 
S[parse-numbers=false, table-format = 3.3] 
S[parse-numbers=false, table-format = 1.4]
}
	\toprule
	{Datos (\si{\cm})} & 
	{Datos2} & 
	{Datos3 (\si{\uW})} & 
	{$n_\text{eff}$} \\
	\midrule
	\DTLforeach*{data}{\dat=ang,\dummy=pow,\cualquierNombre=n,\foo=mic}{% igualar al encabezado de la columna del archivo
		\DTLifnullorempty{\dat}{% Inserta línea si encuentra celda nula o vacía en columna \angulo
			\\[-\normalbaselineskip]% espacio
			\cmidrule(lr){1-4}
		}{  \rceq[0.105]{0.350}{\foo} & 
		    \rceq{8.9403}{\dat} & 
		    \rcgt{\dummy}{45} & 
		    \rcbt{\cualquierNombre}{1.65}{2.06} \\
		} % Inserta los datos en la tabla
	}
	\\[-\normalbaselineskip]
    \bottomrule
\end{tabular}
\label{tab:datos}
\end{table}
%FIN TABLA DE DATOS————————

\lipsum[80-82]

\end{document}
